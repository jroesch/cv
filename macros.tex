\usepackage{color}
% \usepackage[hidelinks]{hyperref}

% Borrowed from: https://tex.stackexchange.com/questions/34881/references-section-in-a-cv

% Define \cvdoublecolumn, which sets its arguments in two columns without any labels
\newcommand{\cvdoublecolumn}[2]{%
\cvitem[0.75em]{}{%
  \begin{minipage}[t]{\listdoubleitemcolumnwidth}#1\end{minipage}%
  \hfill%
  \begin{minipage}[t]{\listdoubleitemcolumnwidth}#2\end{minipage}%
  }%
}

% usage: \cvreference{name}{address line 1}{address line 2}{address line 3}{address line 4}{e-mail address}{phone number}
% Everything but the name is optional
% If \addresssymbol, \emailsymbol or \phonesymbol are specified, they will be used.
% (Per default, \addresssymbol isn't specified, the other two are specified.)
% If you don't like the symbols, remove them from the following code, including the tilde ~ (space).

\newcommand{\cvreference}[7]{%
  \textbf{#1}\newline% Name
  \ifthenelse{\equal{#2}{}}{}{\addresssymbol~#2\newline}%
  \ifthenelse{\equal{#3}{}}{}{#3\newline}%
  \ifthenelse{\equal{#4}{}}{}{#4\newline}%
  \ifthenelse{\equal{#5}{}}{}{#5\newline}%
  \ifthenelse{\equal{#6}{}}{}{\emailsymbol~\texttt{#6}\newline}%
  \ifthenelse{\equal{#7}{}}{}{\phonesymbol~#7}}

\newcommand{\makered}[1]{\textcolor{red}{#1}}
\newcommand{\redbold}[1]{\makered{\textbf{#1}}}
\newcommand{\note}[1]{\redbold{[#1]}}
